A \hyperlink{a00023}{JPPipelineEvent} is supposed to be handled by the \hyperlink{a00019}{JPPipeline} which is attached. Once an event is sent to a \hyperlink{a00019}{JPPipeline}, it is handled by a list of \hyperlink{a00029}{JPPipelineHandler}.

\subsection*{Upstream events and downstream events, and their interpretation}

Every event is either an upstream event or a downstream event. If an event flows forward from the first handler to the last handler in a \hyperlink{a00019}{JPPipeline}, we call it an upstream event and say {\bfseries \char`\"{}an
  event goes upstream.\char`\"{}} If an event flows backward from the last handler to the first handler in a \hyperlink{a00019}{JPPipeline}, we call it a downstream event and say {\bfseries \char`\"{}an event goes downstream.\char`\"{}} (Please refer to the diagram in \hyperlink{a00001}{The Pipeline} for more explanation.) 

When your server receives a \hyperlink{a00006}{message} from a client, the event associated with the received \hyperlink{a00006}{message} is an upstream event. When your server sends a \hyperlink{a00006}{message} or reply to the client, the event associated with the write request is a downstream event. The same rule applies for the client side. If your client sent a request to the server, it means your client triggered a downstream event. If your client received a response from the server, it means your client will be notified with an upstream event. Upstream events are often the result of inbound operations and downstream events are the request for outbound operations.

\subsection*{Cancelling Events}

You can {\bfseries cancel} an {\bfseries downstream} \hyperlink{a00005}{Event} while he is processing sending an \hyperlink{a00010}{JPDefaultCancelEvent} downstream as follow: 
\begin{DoxyCode}
 [pipeline sendDownstream:[JPDefaultCancelEvent init]];
\end{DoxyCode}
 \begin{DoxyNote}{Note}
Instead that \hyperlink{a00001}{The Pipeline} are always an asynchronous I/O operation he process one \hyperlink{a00005}{Event} at a time, that's why you can cancel an event while he is processing or waiting for some answer. 
\end{DoxyNote}


\subsection*{Default Events Messages Types}

\hyperlink{a00006}{Event Message} are some kind of data that is transported by an \hyperlink{a00005}{Event}. \hyperlink{a00006}{Event Message} doesn't have an defined type on the \hyperlink{a00001}{The Pipeline}. The {\bfseries type} of the \hyperlink{a00006}{Event Message} concern to the \hyperlink{a00003}{Handlers}. They use the {\bfseries Message Type} to known if they can handle or not. 

Also the \hyperlink{a00003}{Handlers} usually transform the message on his way through the \hyperlink{a00001}{The Pipeline}. 

{\bfseries JUMP Network} come with a bundled collection of \hyperlink{a00006}{Event Message} that you can use or reuse in your own subclasses.\par
 These are the main groups of this messages:
\begin{DoxyItemize}
\item \hyperlink{a00007}{HTTP Event Messages}
\item \hyperlink{a00008}{JSON-\/RPC Event Messages} 
\end{DoxyItemize}

\subsection*{Additional resources worth reading}

Please refer to the documentation of \hyperlink{a00029}{JPPipelineHandler} and its sub-\/types (\hyperlink{a00035}{JPPipelineUpstreamHandler} for upstream events and \hyperlink{a00021}{JPPipelineDownstreamHandler} for downstream events) to find out how a \hyperlink{a00023}{JPPipelineEvent} is interpreted depending on the type of the handler more in detail. Also, please refer to the \hyperlink{a00019}{JPPipeline} documentation to find out how an event flows in a pipeline. And \hyperlink{a00002}{I$|$O Transporter} to learn about I$|$O transporter and some default implementations. 